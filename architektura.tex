\chapter{Architektura}
\label{cha:architektura}
Na podstawie analizy możliwych rozwiązań sprzętowych w~rozdziale \ref{cha:specyfikacja} oraz przedstawionych poniżej założeń architektury platformy dokonano wyboru urządzeń, które będą stanowiły część sprzętową tworzonego interfejsu.

\section{Założenia architektury sprzętowej}
Aby wybrać odpowiedni dla przygotowywanego prototypu interfejsu zestaw urządzeń, określone zostały założenia i~wymagania, jakie urządzenia powinny spełniać:
\begin{enumerate}
\item Urządzenie powinno być lekkie, aby używanie go przez gracza nie powodowało dyskomfortu, który może wpłynąć negatywnie na jakość odbieranych danych.
\item Dokładność sensorów dostępnych w~urządzeniu powinna być jak największa, aby mieć pewność odczytywanych zachowań użytkownika.
\item Urządzenie powinno być łatwe w~obsłudze, zarówno w~kwestii jego zamontowania, jak i~uruchomienia odczytów. Powinno być także jak najmniej inwazyjne podczas pomiaru, aby zminimalizować dyskomfort użytkownika.
\item Dla wybranego urządzenia powinno być dostępne oprogramowanie umożliwiające odczyt pomiarów w~wybranym środowisku do tworzenia gier. Najlepiej, gdyby rozwiązanie było dostępne w~postaci biblioteki oferowanej przez twórcę sprzętu, lub rozszerzenia silnika, które umożliwi odczyt danych z~urządzenia.
\item Urządzenie powinno mieć możliwość połączenia bezprzewodowego, najlepiej przy pomocy technologii Bluetooth, lub innego protokołu umożliwiającego bezprzewodowy odczyt pomiarów z~urządzenia.
\item Przynajmniej jedno z~urządzeń powinno umożliwiać odczyt pracy serca, w~postaci pomiaru pulsu lub elektrokardiogramu. Pomiar pracy serca jest jedną z~podstawowych metod umożliwiających określenie zmiany w~stanie emocjonalnym użytkownika.
\item Przynajmniej jedno z~urządzeń powinno umożliwiać pomiar ruchów mięśni. Odczytane sygnały mogą być wykorzystane do wpływania na rozgrywkę w~zależności od aktywności mięśniowej użytkownika w~danej partii ciała.
\end{enumerate}


\section{Garmin HRM-Run}
Garmin HRM-Run jest jednym z~wielu dostępnych czujników tętna produkowanych przez firmę Garmin. Urządzenie zbudowane jest z~modułu zawierającego całą elektronikę odpowiedzialną za interpretację oraz wysyłanie odczytywanych danych. Moduł zasilany jest baterią CR2032 i~zamontowany jest na elastycznej opasce, umożliwiającej zamontowanie opaski na klatce piersiowej. Sama opaska zawiera dwie elektrody, które odpowiadają za odczyt sygnałów fizjologicznych, które następnie są przekazywane w~formie informacji na temat pracy serca. 

Urządzenie, poza podstawowym pomiarem liczby uderzeń na minutę, pozwala na odczyt dodatkowych informacji na temat zmienności rytmu zatokowego w~postaci odstępów czasowych pomiędzy kolejnymi tzw. załamkami R, czyli szczytami kompleksów QRS. Na podstawie tej informacji można określić stan zdrowotny użytkownika lub zmiany jego stanu emocjonalnego. Dla przykładu nagłe spadki w~zmienności rytmu zatokowego mogą sygnalizować możliwość odczuwania stresu przez użytkownika. HRM-Run umożliwia także odczyt pomiarów dynamiki biegu takich jak liczba kroków na minutę, czas kontaktu z~podłożem czy długość kroku. Ze względu na charakter pracy, która skupia się głównie na tematyce gier komputerowych, pomiary te nie były brane pod uwagę. 

Urządzenie HRM-Run zostało wybrane, ponieważ stanowi pewien kompromis pomiędzy dokładnością odczytów a~rozmiarami i~wygodą użytkowania. W~porównaniu do inteligentnych opasek, które charakteryzują się najwyższą wygodą użytkowania spośród omówionych w~rozdziale \ref{cha:specyfikacja} urządzeń umożliwiających pomiar tętna, Garmin HRM-Run pozwala na o~wiele dokładniejsze odczyty, nie powodując jednocześnie dyskomfortu podczas użytkowania. Jest to możliwe dzięki wykorzystaniu elastycznej opaski ze wbudowanymi suchymi elektrodami, które w~przeciwieństwie do fotopletyzmografu odpowiadającego za optyczny odczyt tętna w~inteligentnych opaskach, odbierają sygnały elektryczne bezpośrednio z~ciała użytkownika, które są następnie interpretowane do danych w~postaci wykrycia kolejnych uderzeń serca. Opaski charakteryzują się także dużymi opóźnieniami w~stosunku do aktualnego tętna użytkownika, co w~przypadku rozpoznawania emocji w~sposób ciągły całkowicie eliminuje je jako akceptowalne urządzenia. Jednocześnie, elastyczna opaska, do której przymocowany jest moduł, w~bardzo prosty sposób jest zakładana przez użytkownika na klatce piersiowej, Takie umiejscowienie i~brak kabli łączących elektrody z~głównym modułem sprawia, że użytkownik nie czuje dyskomfortu podczas noszenia urządzenia. W~podobny sposób Garmin HRM-Run można porównać z~omawianym wcześniej BITalino (r)evolution kit. W~tym przypadku można zauważyć sytuację odwrotną, niż przy porównaniu z~inteligentnymi opaskami. BITalino oferuje bardzo dokładny odczyt pracy serca, nie tylko w~formie liczby uderzeń na minutę, ale pełnego elektrokardiogramu, który umożliwia o~wiele szerszą analizę pracy serca. Niestety, odczyt wymaga zamocowania przyklejanych elektrod, które połączone są przewodami z~sensorem, a~następnie z~samym BITalino. Taki sposób montażu może wpłynąć negatywnie na komfort podczas gry, co nie wystąpi w~przypadku urządzenia HRM-Run.

Innym aspektem, który zadecydował o~wyborze tego urządzenia, jest obsługa dwóch protokołów bezprzewodowego przesyłania danych. Poza standardowym połączeniem dzięki technologii Bluetooth HRM-Run wykorzystuje także protokół ANT+ rozwijany przez firmę Garmin. Jego główną zaletą jest brak ograniczeń co do ilości urządzeń, które mogą odczytywać dane z~czujnika. W~przeciwieństwie do technologii Bluetooth, która ogranicza połączenie do jednego, lub w~przypadku Bluetooth Smart do dwóch urządzeń, pomiar przy pomocy protokołu ANT+ umożliwia jednoczesne odczytywanie danych w~grze i~monitorowanie pracy serca przy pomocy innych urządzeń lub aplikacji korzystających z~tego protokołu. 

\begin{figure}
	\centering
	\includegraphics[width=0.5\linewidth]{images/garmin_hrm_placement.png}
	\caption{Miejsce montażu czujnika tętna Garmin HRM-Run, źródło:~\cite{garmin_manual}}
	\label{fig:garmin_placement}
\end{figure}

\section{BITalino Revolution Kit}

\begin{figure}
	\centering
	\includegraphics[height=0.3\textheight]{images/bitalino_placement.jpg}
	\caption{Sposób przypięcia elektrod dla sensora EMG,  kolor czerwony oznacza elektrodę dodatnią, czarny ujemną, a~szary elektrodę referencyjną}
	\label{fig:bitalino_placement}
\end{figure}

\section{Dualshock 4}
krótko o~urządzeniu, wykorzystanie akcelerometru do odczytu pobudzenia gracza
