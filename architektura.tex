\chapter{Architektura}
\label{cha:architektura}
Na podstawie analizy możliwych rozwiązań sprzętowych w~rozdziale~\ref{cha:specyfikacja} oraz przedstawionych poniżej założeń architektury platformy dokonano wyboru urządzeń, które będą stanowiły część sprzętową tworzonego interfejsu. W~tym rozdziale przedstawiono ich specyfikacje oraz argumenty, które przemawiają za wyborem każdego z~nich. Opisano także sposób montażu sprzętu.

\section{Założenia architektury sprzętowej}
Aby wybrać odpowiedni dla przygotowywanego prototypu interfejsu zestaw urządzeń, określone zostały założenia i~wymagania, jakie powinny one spełniać:
\begin{enumerate}
	\item Platforma sprzętowa powinna być lekka, aby używanie jej przez gracza nie powodowało dyskomfortu, który może wpłynąć negatywnie na jakość odbieranych danych.
	\item Dokładność dostępnych sensorów powinna być jak największa, aby mieć pewność odczytywanych zachowań użytkownika.
	\item Urządzenie powinno być łatwe w~obsłudze, zarówno w~kwestii jego zamontowania, jak i~uruchomienia odczytów. Jednocześnie powinno charakteryzować się jak najmniejszą inwazyjnością podczas pomiaru, aby zminimalizować dyskomfort użytkownika.
	\item Dla wybranego sprzętu powinno być dostępne oprogramowanie umożliwiające odczyt sygnałów w~wybranym środowisku do tworzenia gier. Najlepiej, gdyby rozwiązanie było dostępne w~postaci biblioteki oferowanej przez twórcę narzędzia, lub rozszerzenia silnika, które umożliwi odbieranie danych.
	\item Urządzenie powinno mieć możliwość połączenia bezprzewodowego, najlepiej przy pomocy technologii Bluetooth lub innego protokołu umożliwiającego bezprzewodowy przesył danych.
	\item Przynajmniej jedno z~urządzeń powinno umożliwiać odczyt pracy serca w~postaci pulsu lub sygnału z elektrokardiogramu. Pomiar tętna jest jedną z~podstawowych metod umożliwiających określenie zmian w~stanie emocjonalnym użytkownika.
	\item Przynajmniej jedno z~urządzeń powinno umożliwiać odczytywanie ruchów mięśni. Uzyskane sygnały mogą być wykorzystane do wpływania na rozgrywkę w~zależności od aktywności mięśniowej użytkownika w~danej partii ciała.
\end{enumerate}


\section{Garmin HRM-Run}
Garmin HRM-Run jest jednym z~wielu dostępnych czujników tętna produkowanych przez firmę Garmin. Urządzenie zbudowane jest z~modułu zawierającego elektronikę odpowiedzialną za interpretację oraz wysyłanie odczytywanych danych. Zasilany jest on baterią CR2032 i~zamontowany jest na elastycznej opasce, umożliwiającej zamontowanie opaski na klatce piersiowej. Sama opaska zawiera dwie elektrody odpowiadające za odczyt sygnałów fizjologicznych, które następnie są przekazywane w~formie informacji na temat pracy serca. 

Czujnik, poza podstawowym pomiarem liczby uderzeń na minutę, pozwala na pozyskiwanie dodatkowych informacji na temat zmienności rytmu zatokowego w~postaci odstępów czasowych pomiędzy kolejnymi tzw. załamkami R, czyli szczytami kompleksów QRS. Na podstawie tej informacji można określić stan zdrowotny użytkownika lub zmiany jego stanu emocjonalnego. Dla przykładu nagłe spadki w~zmienności rytmu zatokowego mogą sygnalizować możliwość odczuwania stresu przez użytkownika. HRM-Run umożliwia także zbieranie danych na temat dynamiki biegu, takich jak liczba kroków na minutę, czas kontaktu z~podłożem czy długość kroku. Ze względu na charakter pracy, która skupia się głównie na tematyce gier komputerowych, pomiary te nie były brane pod uwagę. 

Urządzenie HRM-Run zostało wybrane, ponieważ stanowi pewien kompromis pomiędzy dokładnością odczytów a~rozmiarami i~wygodą użytkowania. W~porównaniu do inteligentnych opasek, charakteryzujących się najwyższym poziomem komfortu spośród omówionych w~rozdziale \ref{cha:specyfikacja} platform sprzętowych umożliwiających pomiar tętna, Garmin HRM-Run pozwala na o~wiele dokładniejsze odczyty, nie powodując przy tym dyskomfortu podczas użytkowania. Jest to możliwe dzięki wykorzystaniu elastycznej opaski ze wbudowanymi suchymi elektrodami, które, w~przeciwieństwie do fotopletyzmografu odpowiadającego za optyczny odczyt tętna w~inteligentnych opaskach, odbierają sygnały elektryczne bezpośrednio z~ciała użytkownika. Informacje te są następnie interpretowane do danych w~postaci wykrycia kolejnych uderzeń serca. Opaski natomiast charakteryzują się dużymi opóźnieniami w~stosunku do aktualnego tętna użytkownika, co w~przypadku rozpoznawania emocji w~sposób ciągły całkowicie eliminuje je jako akceptowalne rozwiązanie. Jednocześnie elastyczna opaska, do której przymocowany jest moduł, w~bardzo prosty sposób jest zakładana przez użytkownika na klatce piersiowej. Takie umiejscowienie i~brak kabli łączących elektrody z~głównym modułem sprawia, że użytkownik nie czuje dyskomfortu podczas noszenia urządzenia. W~podobny sposób Garmin HRM-Run można porównać z~omawianym wcześniej BITalino (r)evolution kit. W~tym przypadku można zauważyć sytuację odwrotną niż przy porównaniu z~inteligentnymi opaskami. BITalino oferuje bardzo dokładny pomiar pracy serca, nie tylko w~formie liczby uderzeń na minutę, ale także pełnego elektrokardiogramu umożliwiającego o~wiele szerszą analizę pracy serca. Niestety, odczyt wymaga zamocowania przyklejanych elektrod, które połączone są przewodami z~sensorem, a~następnie z~samym BITalino. Taki sposób montażu może wpłynąć negatywnie na komfort podczas gry, co nie wystąpi w~przypadku urządzenia HRM-Run.

Innym aspektem decydującym o~wyborze tej platformy sprzętowej, jest obsługa dwóch protokołów bezprzewodowego przesyłania danych. Poza standardowym połączeniem dzięki technologii Bluetooth HRM-Run wykorzystuje także protokół ANT+ rozwijany przez firmę Garmin. Jego główną zaletą jest brak ograniczeń co do ilości urządzeń mogących odbierać dane z~czujnika. W~przeciwieństwie do technologii Bluetooth, która ogranicza połączenie do jednego, lub w~przypadku Bluetooth Smart dwóch urządzeń, pomiar przy pomocy protokołu ANT+ umożliwia jednoczesne odczytywanie danych w~grze i~monitorowanie pracy serca przy pomocy innych narzędzi korzystających z~tego protokołu. Twórcy ANT+ udostępniają szeroką dokumentację oraz biblioteki umożliwiające odczyt w~wielu językach programowania, od C++, C\#, aż po dedykowane implementacje dla systemu Android.

\begin{figure}
	\centering
	\includegraphics[width=0.5\linewidth]{images/garmin_hrm_placement.png}
	\caption{Miejsce montażu czujnika tętna Garmin HRM-Run, źródło:~\cite{garmin_manual}}
	\label{fig:garmin_placement}
\end{figure}

\section{BITalino (r)evolution kit}
BITalino (r)evolution Plugged Kit BT jest urządzeniem produkowanym przez firmę PLUX Wireless Biosignals przeznaczonym do tworzenia platform, w~których zawarte są elementy wymagające informacji na temat sygnałów fizjologicznych. Główny element narzędzia stanowi płytka zbudowana z~następujących części~\cite{bitalino_documentation}:
\begin{itemize}
	\item 10~złącz UC-E6, w~ramach których można wyróżnić: 6~wejść analogowych, 1~wejście cyfrowe, 1~wyjście cyfrowe, 1~złącze mogące pracować w~trybie wejścia lub wyjścia cyfrowego, oraz 1~złącze PWM, do którego może zostać podłączony przetwornik DAC lub dioda LED. Złącza są podzielone na 2~grupy w~oddzielnych segmentach.
	\item Mikrokontroler z~mikroprocesorem oraz stykami umożliwiającymi połączenie każdego z~segmentów w~sposób inny niż standardowy
	\item Segment zasilający, zawierający włącznik, gniazdo ładownia w~formie złącza micro-USB oraz złącze JST, do którego podłączana jest bateria 3.7V zasilająca całą płytkę.
	\item Moduł Bluetooth odpowiadający za przesył danych z~płytki.
\end{itemize}

W ramach przygotowywanego interfejsu, z~dostępnych sensorów opisanych w~rozdziale \ref{cha:specyfikacja} wybrany został jedynie moduł zawierający elektromiogram mierzący napięcie mięśni podskórnych. Głównym powodem takiego wyboru są dwie główne wady urządzenia dotyczące komfortu użytkowania. Są to przewody łączące płytkę z~sensorem oraz elektrody mocowane z drugiej strony przewodów. Choć z~perspektywy eksperymentów naukowych użycie elektrod żelowych przyklejanych do skóry nie jest problemem, to w~przypadku środowiska gier komputerowych standardowy użytkownik może odczuwać dyskomfort lub w~ogóle zrezygnować z~używania urządzenia ze względu na jego inwazyjność.

Głównym powodem wyboru BITalino oraz samego modułu EMG jest wykorzystanie sygnałów fizjologicznych do kontrolowania gry w~świadomy sposób. Odczyt zmian napięcia mięśni podskórnych może pozwolić na wykrycie, kiedy i~jak mocno są one używane, co może następnie zostać wykorzystane do zaimplementowania mechanik w~grze uruchamianych na podstawie konkretnych reakcji mięśniowych. Kolejnym powodem wykorzystania platformy sprzętowej jest możliwość sprawdzenia, w~jakim stopniu użytkownik tak naprawdę odczuwa dyskomfort podczas używania elektrod przyklejanych do skóry. Pozwoli to stwierdzić, czy urządzenia takie jak BITalino, które umożliwiają wykonanie dokładnych pomiarów dzięki wykorzystaniu elementów bliższych rozwiązaniom medycznym, mogłyby przyjąć się jako stały element stanowiska do gier.

Ponieważ odczyty z~elektromiogramu mają posłużyć jako pewien element kontroli nad grą, jako miejsce, do którego podłączone będą elektrody, wybrano przedramię, ponieważ ruch mięśni znajdujących się w~tamtej części ciała może być precyzyjny. Dzięki temu użytkownik w~prosty sposób, poprzez ruchy nadgarstka lub dłoni, może kontrolować napięcie mięśni w~trakcie rozgrywki. Sposób montażu elektrod został przedstawiony na rysunku~\ref{fig:bitalino_placement}. Elektrody czerwona i~czarna powinny znajdować się na wewnętrznej części przedramienia, równolegle do niego, stosunkowo blisko siebie. Elektroda referencyjna, w~BITalino oznaczona kolorem białym, powinna znajdować się na wystającej kości łokciowej.
\begin{figure}
	\centering
	\includegraphics[height=0.3\textheight]{images/bitalino_placement.jpg}
	\caption{Sposób przypięcia elektrod dla sensora EMG,  kolor czerwony oznacza elektrodę dodatnią, czarny ujemną, a~szary elektrodę referencyjną}
	\label{fig:bitalino_placement}
\end{figure}

\section{DualShock 4}
DualShock 4~jest urządzeniem produkowanym przez firmę Sony, stworzonym początkowo jako kontroler do konsoli PlayStation 4. Na przyciski dostępne na jego powierzchni składają się~\cite{dualshock_specification}:
\begin{itemize}
	\item 2~gałki analogowe, gdzie dla każdej z~nich odczyty interpretowane są jako ciągły sygnał wejściowy przedstawiony w~dwóch wymiarach.
	\item Pad kierunkowy (ang. \textit{d-pad}) stanowiący cyfrowy odpowiednik gałki analogowej. Lewy i~prawy przycisk odpowiadają wartościom granicznym osi horyzontalnej, górny i~dolny są przypisane w~podobny sposób do osi wertykalnej.
	\item 9~przycisków cyfrowych, na które składają się: 4~przyciski akcji (trójkąt, kwadrat, krzyżyk, kółko), L3 i~R3 zamontowane pod gałkami analogowymi oraz przyciski PS, SHARE i~OPTIONS, domyślnie służące do zarządzania funkcjonalnościami konsoli.
	\item 2~przyciski L1 i~R1, dla których wykrycie wciśnięcia opiera się na sile nacisku na przycisk.
	\item 2~analogowe spusty L2 i~R2, których wciśnięcie jest odczytywane jako sygnał w~określonym zakresie wartości.
	\item Dwupunktowy panel dotykowy, posiadający także możliwość kliknięcia go.
\end{itemize}
Innymi elementami widocznymi na urządzeniu są wbudowany głośnik, wejście słuchawkowe Jack 3.5mm, umieszczony na froncie pasek świetlny RGB oraz złącze micro-USB służące do ładowania i~podłączenia kontrolera w~przypadku braku połączenia Bluetooth. Kontroler, wraz ze wszystkimi opisanymi powyżej elementami, został przedstawiony na rysunku~\ref{fig:hardware}.

Poza widocznymi elementami Dualshock 4 posiada także wbudowane moduły wibrujące, żyroskop oraz akcelerometr. To właśnie dwa ostatnie z~wymienionych elementów wpłynęły na wybór urządzenia jako części przygotowywanego interfejsu. W~porównaniu do innych dostępnych kontrolerów, poza klasycznym wykorzystaniem narzędzia jako sposobu kierowania postacią w grze, akcelerometr wbudowany w~DualShocka może posłużyć jako źródło danych, które w~pośredni sposób mogą określać pewne elementy stanu emocjonalnego użytkownika. Bardziej pobudzony gracz, może zupełnie nieświadomie poruszać kontrolerem, natomiast brak zmian w~odczytach może sugerować, że użytkownik jest skupiony lub znudzony. W~porównaniu do określania ilości kliknięć na myszy i~klawiaturze w~danym czasie, wykorzystanie akcelerometru jest lepszym rozwiązaniem, ponieważ ruchy dłoni gracza trzymającego kontroler są instynktowne w~sytuacjach wywołujących pobudzenie. Sceny wywołujące nagły strach, mogą doprowadzić do ruchu całego ciała, natomiast sytuacje powodujące uczucie złości lub desperacji mogą wywołać subtelne przesunięcie pozycji rąk na kontrolerze, co wywoła zmianę wartości na akcelerometrze. Interpretacja tych danych oraz odczytów pracy serca pozwoli na dokładniejsze określenie, jaką emocję odczuwa gracz w~danym momencie.

Problemem, który warto zaznaczyć w~przypadku DualShocka, jest ograniczenie dostępności oprogramowania pozwalającego odczytywać pomiary z~akcelerometru. Z~powodu licencji narzuconych przez producenta, większość dodatków i~bibliotek wykorzystywanych w~silnikach do gier jest dostępnych wyłączenie w~wersji płatnej, a~oficjalne oprogramowanie od twórców jest udostępniane wyłącznie osobom zatwierdzonym przez Sony jako projektanci gier na konsolę PlayStation 4. Jednak pomimo tego ograniczenia, zdecydowano się na wybór Dualshocka 4 jako części przygotowywanego interfejsu.

\begin{figure}
	\centering
	\includegraphics[width=0.7\linewidth]{images/hardware.jpg}
	\caption{Moduły sprzętowe wykorzystywane w~platformie. Od lewej: kontroler DualShock 4, czujnik tętna Garmin HRM-Run, BITalino z~modułem EMG}
	\label{fig:hardware}
\end{figure}

Na rysunku~\ref{fig:hardware} przedstawiony został pełen zestaw urządzeń wykorzystanych w~przygotowywanym interfejsie. Pozyskane z~nich pomiary zostaną wykorzystane do rozpoznania stanu emocjonalnego gracza, a~także do kontrolowania mechanik wykorzystywanych w~grze. Dokładny opis, w~jaki sposób dane będą odczytywane oraz jak będzie wyglądał model do predykcji emocji użytkownika, zostaną opisane w~rozdziałach~\ref{cha:predykcja} i~\ref{cha:implementacja}.