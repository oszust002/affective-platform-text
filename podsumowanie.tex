\chapter{Podsumowanie}
\label{cha:podsumowanie}
\section{Wnioski}
Niniejsza praca miała na celu zaprojektowanie i~przygotowanie interfejsu składającego się ze zbioru platform sprzętowych umożliwiających pomiar sygnałów fizjologicznych i~zachowań użytkownika oraz modułu programistycznego w~wybranym środowisku do tworzenia gier, którego zadaniem jest określenie stanu emocjonalnego i~zachowań użytkownika. Rozwiązanie zostało zrealizowane w~kilku etapach.

Pracę rozpoczęto od przeglądu literatury o~tematyce emocji i~programowania afektywnego. Szczególną uwagę poświęcono modelom reprezentacji stanów emocjonalnych i~pojęciu pętli afektywnej. Odwołano się także do aktualnie istniejących gier, które realizują założenia programowania afektywnego lub wykorzystują sygnały fizjologiczne do wpływania na rozgrywkę.

Kolejnym krokiem była analiza dostępnych rozwiązań w~zakresie platform sprzętowych do pomiaru sygnałów fizjologicznych, możliwych mechanizmów wnioskowania oraz narzędzi wykorzystywanych do budowy gier komputerowych. Opisane zostały wady i~zalety wyboru poszczególnych rozwiązań sprzętowych, działanie wybranych metod wnioskowania oraz możliwości kilku dostępnych darmowych silników do tworzenia gier. Podczas analizy rozwiązań sprzętowych zwrócono uwagę na wyższość urządzeń nasobnych, które oferowały wysoki komfort użytkowania, jednocześnie pozwalając na dokładny odczyt sygnałów fizjologicznych. Zwrócono także uwagę na urządzenia, które oferowały pomiary pośrednie, takie jak odczyty z~kontrolerów do gier, umożliwiające określenie stanu użytkownika.

Następnie przeprowadzone zostało przygotowanie części sprzętowej projektu. Określone zostały podstawowe założenia wymagane do spełnienia przez urządzenia wchodzące w~skład interfejsu. Najważniejszym aspektem było uzyskanie jak najdokładniejszych pomiarów, nie powodując przy tym wzrostu dyskomfortu w~czasie użytkowania. Dla każdego z~urządzeń przedstawione zostały argumenty uzasadniające ich wybór. Zwrócono uwagę na dostępne sposoby komunikacji, a~także sposób montażu każdego z~urządzeń.

Część programistyczną projektu rozpoczęto od przygotowania modelu uczenia maszynowego do rozpoznawania emocji. Wybrane zostały zbiory danych uczących, które następnie poddano wstępnemu przetworzeniu. Określono także cechy bazujące na pomiarach pracy serca oraz zbiór klas rozpoznawanych przez model, które stanowią reprezentację poszczególnych stanów emocjonalnych. Następnie przetestowano modele oparte na kilku algorytmach uczenia maszynowego i~wybrano ostatecznie klasyfikator ekstremalnych drzew losowych charakteryzujący się najwyższą skutecznością.

Ostatnią częścią przygotowaną w~ramach niniejszej pracy była implementacja głównego elementu, którym był moduł programistyczny komunikujący się z~wybranymi urządzeniami i~określający stan emocjonalny użytkownika. Równocześnie została także stworzona gra komputerowa, która wykorzystywała interfejs do obsługi zmian mechanik rozgrywki w~zależności od zachowań i~emocji gracza. 

Stworzona gra została następnie wykorzystana w~celu ewaluacji zaimplementowanego rozwiązania. Głównymi kryteriami oceny były logi emocji określonych przez interfejs oraz kwestionariusze wypełnione przez uczestników badania po jego zakończeniu. Analizując wyniki dotyczące zauważenia zmian w~rozgrywce w~wersji gry zawierającej interfejs do określania stanu emocjonalnego gracza, wygody użytkowania części sprzętowej oraz satysfakcji osób badanych z~rozgrywki można stwierdzić, że moduł stworzony w~ramach niniejszej pracy magisterskiej realizuje główny cel projektu. Taka ocena może wskazywać także na to, że wprowadzenie mechanizmów opartych o~emocje na rynek komercyjny gier komputerowych ma nadzieję spotkać się z~pozytywnym odzewem branży oraz graczy.

\section{Propozycje przyszłych prac}
Choć stworzony interfejs spełnia wymagania określone w~celach niniejszej pracy magisterskiej, możliwe jest rozszerzenie poszczególnych jego elementów. Można rozważyć dodanie nowych urządzeń, pozwalających na pobieranie innych sygnałów fizjologicznych, takich jak reakcja elektrodermalna, czy pomiary zwracane przez elektroencefalograf. Ponieważ struktura interfejsu jest modułowa, rozwiązanie to wymagałoby przygotowania skryptów do komunikacji i~interpretacji nowych sygnałów. 

Elementem wartym dopracowania jest także model rozpoznający emocje. Poprzez wprowadzenie nowych sygnałów możliwe byłoby przygotowanie zbioru o~wyższej skuteczności. Jednocześnie rozwiązaniem wartym uwagi jest wykorzystanie sieci neuronowej zamiast klasyfikatorów. Możliwe, że taka zmiana podejścia podczas projektowania modelu do rozpoznawania emocji pozwoli uzyskać rezultaty znacznie wyższe od uzyskanych w~ramach niniejszej pracy magisterskiej. Jednocześnie w~ramach modelu mogłyby też zostać uwzględnione odczyty bazowe zbierane w~trakcie fazy kalibracji. Pozwoliłoby to na porównanie sygnałów fizjologicznych zbieranych w~czasie gry z~wartościami spoczynkowymi. 

Ponadto, przygotowany moduł zawiera także elementy wymagające poprawy. Poza wspomnianym wcześniej modelem do rozpoznawania emocji należałoby zabezpieczyć także moment weryfikacji zwracanych emocji przez odczyty z~akcelerometru w~taki sposób, aby nie obniżał nadmiernie wartości \textit{arousal} w~przypadku gdy użytkownik nie porusza kontrolerem w~czasie gry.


