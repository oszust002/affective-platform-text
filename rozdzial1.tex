\chapter{Wstęp}
\label{cha:wstep}
W ciągu ostatnich lat pojęcie sztucznej inteligencji przestało być tylko fenomenem, który dla większości społeczeństwa istniał pod postacią filmów lub powieści dotyczących maszyn posiadających świadomość. Od tamtego momentu człowiek zaczął znajdować zastosowanie sztucznej inteligencji w coraz większej liczbie dziedzin. Od maszyn przetwarzających w sposób automatyczny ogromne ilości informacji, aż po systemy gromadzące dane dotyczące użytkowników i~na ich podstawie generują reguły, dzięki którym możliwe jest znalezienie rozwiązań i~porad dopasowanych do danego użytkownika.

Co jest w tym wszystkim najistotniejsze, to fakt, że człowiek otacza się sztuczną inteligencją, nawet tego nie zauważając. Systemy rekomendujące, dopasowujące produkty z każdej tematyki do preferencji użytkowników~\cite{Gomez-Uribe:2015:NRS:2869770.2843948}, urządzenia nasobne monitorujące nasz stan zdrowia dzięki pomiarom parametrów życiowych i~zbieraniu informacji na temat naszych nawyków~\cite{wearable_computing_amft}, czy coraz popularniejsze systemy autonomicznej jazdy~\cite{dikmen_tesla_autopilot}. Sztuczna inteligencja z dnia na dzień coraz bardziej wnika w każdy aspekt życia człowieka. 

Jednym z elementów, który odgrywa ważną rolę w życiu człowieka, a który coraz mocniej oparty jest o sztuczną inteligencję, są gry komputerowe.  Choć początkowo były one traktowane wyłącznie jako rozrywka, to dziś coraz częściej są określane nawet jako pewna forma nauki konkretnych umiejętności~\cite{oberdorfer_develop_your_strengths_by_gaming}. Przykładem mogą być tutaj gatunek gier strategicznych, które uczą taktyki i~zarządzania, a także gry logiczne pozwalające na rozwój logicznego myślenia. Warto tutaj wspomnieć także o grach poważnych~\cite{serious_games_michael_chen}, które nie skupiają się na aspektach rozrywkowych, a bardziej są formą edukacji i~szkoleń przedstawionych w formie interaktywnej symulacji.

Pojęciem, które jest coraz szerzej widoczne w kontekście sztucznej inteligencji związanej przede wszystkim z tematyką komunikacji człowiek-komputer jest informatyka afektywna. Chociaż samo pojęcie istnieje już od ponad 20 lat~\cite{Picard:1997:AC:265013}, to dopiero w ciągu kilku ostatnich badania w tej tematyce stały się popularne~\cite{gartner_hype_cycles_2018}. Początkowo główną ideą tej dziedziny były systemy zbierające i~analizujące dane na temat stanów emocjonalnych użytkowników. Ponieważ emocje nie mogą być kontrolowane poprzez działania człowieka, a jednocześnie można je opisać między innymi przy pomocy zmian fizjologicznych w ludzkim ciele, wykorzystanie informatyki afektywnej w systemach inteligentnych takich jak aplikacje rekomendujące czy systemy badające stan zdrowotny użytkowników pozwala na zwiększenie ich skuteczności działania.

W podobny sposób powstała próba powiązania dziedziny informatyki afektywnej z grami komputerowymi, tworząc nowy rodzaj gier, nazywanych grami afektywnymi. Główną ich ideą jest pomiar stanów emocjonalnych wywoływanych na użytkowniku w trakcie rozgrywki oraz dostosowywanie gry w czasie rzeczywistym do odczytanych reakcji gracza, tak aby zwiększyć doznania płynące z gry~\cite{kotsia_affective_gaming}. Dzięki temu każda gra może zostać w pewien sposób spersonalizowana na podstawie indywidualnych cech użytkownika. 

Celem pracy jest opracowanie prototypu dwuczęściowego interfejsu umożliwiającego pomiar sygnałów pozwalających na określenie zmian stanów emocjonalnych gracza. Interfejs ma posłużyć do opracowania prototypów gier zawierających pętlę afektywną. W skład interfejsu będą wchodzić:
\begin{itemize}
	\item zbiór urządzeń umożliwiających pomiary sygnałów wykorzystanych do określenia zmian emocji gracza
	\item moduł przygotowany w środowisku do tworzenia gier, który na podstawie zgromadzonych z urządzeń pomiarowych sygnałów będzie określał stan emocjonalny i~zachowania użytkownika
\end{itemize}
Ważnym elementem pracy jest stworzenie gry zawierającej pętlę afektywną~\cite{affective_loop_experiences}, która będzie wykorzystywała przygotowany interfejs. Po wykryciu zachowań oraz zmian stanów emocjonalnych użytkownika, stan gry jest aktualizowany.

Niniejsza praca składa się z 8 rozdziałów. W rozdziale \ref{cha:affectiveComputing} zostały przedstawione podstawy teoretyczne dotyczące informatyki afektywnej. Szczególny nacisk położono na tematykę gier afektywnych, przedstawiając wybrane istniejące rozwiązania, problemy i~kierunki badań z tego zakresu. Rozdział \ref{cha:specyfikacja} zawiera przedstawienie oraz analizę dostępnych sprzętowych platform pomiarowych, możliwych mechanizmów wnioskowania oraz narzędzi wykorzystywanych do budowy gier komputerowych. Ważnym elementem jest przedstawienie wad i~zalet każdego z rozwiązań w kontekście tematyki pracy. Rozdział \ref{cha:architektura} jest podsumowaniem analizy platform sprzętowych z poprzedniego rozdziału. Przedstawione tu zostały podstawowe założenia, jakie powinny być spełnione przez wybraną grupę urządzeń pomiarowych, a także definiuje sprzęt wybrany podczas końcowej implementacji. W rozdziale \ref{cha:predykcja} opisany został proces budowy modelu do rozpoznawania emocji. Omówione zostały wybrane zbiory danych, sposób ich przetwarzania, oraz budowa i~wybór końcowego modelu na podstawie działania z dostępnymi danymi. Rozdział \ref{cha:implementacja} jest jedną z najistotniejszych części pracy. Przedstawiono w nim proces implementacji utworzonej gry komputerowej z pętlą afektywną. Skupiono się na opisie interfejsu łączącego rozwiązania wybrane w poprzednich rozdziałach i~sposobie jego wykorzystania wewnątrz gry. Następnie przedstawione zostały mechaniki pokazujące, w jaki sposób przygotowany moduł może wpłynąć na rozgrywkę tak, by sprzężenie zwrotne mogło zostać zamknięte. W rozdziale \ref{cha:badania} opisany został sposób ewaluacji stworzonego rozwiązania oraz charakterystyka i~proces przeprowadzonych eksperymentów. Ostatni rozdział stanowi podsumowanie niniejszej pracy. Zawiera wnioski dotyczące przygotowanego projektu, jego mocne i~słabe strony. Opisane zostały także możliwe kierunki dalszego rozwoju projektu. 
